\documentclass[12pt]{report}
\usepackage[utf8]{inputenc}
\usepackage[russian]{babel}
%\usepackage[14pt]{extsizes}
\usepackage{listings}
\usepackage{graphicx}
\usepackage{amsmath,amsfonts,amssymb,amsthm,mathtools} 
\usepackage{pgfplots}
\usepackage{filecontents}
\usepackage{float}
\usepackage{comment}
\usepackage{indentfirst}
\usepackage{eucal}
\usepackage{enumitem}
%s\documentclass[openany]{book}
\frenchspacing

\usepackage{indentfirst} % Красная строка

\usetikzlibrary{datavisualization}
\usetikzlibrary{datavisualization.formats.functions}

\usepackage{amsmath}


% Для листинга кода:
\lstset{ %
	language=c,                 % выбор языка для подсветки (здесь это С)
	basicstyle=\small\sffamily, % размер и начертание шрифта для подсветки кода
	numbers=left,               % где поставить нумерацию строк (слева\справа)
	numberstyle=\tiny,           % размер шрифта для номеров строк
	stepnumber=1,                   % размер шага между двумя номерами строк
	numbersep=5pt,                % как далеко отстоят номера строк от подсвечиваемого кода
	showspaces=false,            % показывать или нет пробелы специальными отступами
	showstringspaces=false,      % показывать или нет пробелы в строках
	showtabs=false,             % показывать или нет табуляцию в строках
	frame=single,              % рисовать рамку вокруг кода
	tabsize=2,                 % размер табуляции по умолчанию равен 2 пробелам
	captionpos=t,              % позиция заголовка вверху [t] или внизу [b] 
	breaklines=true,           % автоматически переносить строки (да\нет)
	breakatwhitespace=false, % переносить строки только если есть пробел
	escapeinside={\#*}{*)}   % если нужно добавить комментарии в коде
}


\usepackage[left=2cm,right=2cm, top=2cm,bottom=2cm,bindingoffset=0cm]{geometry}
% Для измененных титулов глав:
\usepackage{titlesec, blindtext, color} % подключаем нужные пакеты
\definecolor{gray75}{gray}{0.75} % определяем цвет
\newcommand{\hsp}{\hspace{20pt}} % длина линии в 20pt
% titleformat определяет стиль
\titleformat{\chapter}[hang]{\Huge\bfseries}{\thechapter\hsp\textcolor{gray75}{|}\hsp}{0pt}{\Huge\bfseries}


% plot
\usepackage{pgfplots}
\usepackage{filecontents}
\usetikzlibrary{datavisualization}
\usetikzlibrary{datavisualization.formats.functions}

\begin{document}
	%\def\chaptername{} % убирает "Глава"
	\thispagestyle{empty}
	\begin{titlepage}
		\noindent \begin{minipage}{0.15\textwidth}
			\includegraphics[width=\linewidth]{img/b_logo}
		\end{minipage}
		\noindent\begin{minipage}{0.9\textwidth}\centering
			\textbf{Министерство науки и высшего образования Российской Федерации}\\
			\textbf{Федеральное государственное бюджетное образовательное учреждение высшего образования}\\
			\textbf{~~~«Московский государственный технический университет имени Н.Э.~Баумана}\\
			\textbf{(национальный исследовательский университет)»}\\
			\textbf{(МГТУ им. Н.Э.~Баумана)}
		\end{minipage}
		
		\noindent\rule{18cm}{3pt}
		\newline\newline
		\noindent ФАКУЛЬТЕТ $\underline{\text{«Информатика и системы управления»}}$ \newline\newline
		\noindent КАФЕДРА $\underline{\text{«Программное обеспечение ЭВМ и информационные технологии»}}$\newline\newline\newline\newline\newline
		
		\begin{center}
			\noindent\begin{minipage}{1.1\textwidth}\centering
				\Large\textbf{  Отчет по лабораторной работе №2}\newline
				\textbf{по дисциплине <<Математическая статистика>>}\newline\newline\newline\newline
			\end{minipage}
		\end{center}
		
		\noindent\textbf{Тема} $\underline{\text{Интервальные оценки~~~~~~}}$\newline\newline
		\noindent\textbf{Студент} $\underline{\text{Романов А.В.~~~~~~~~~~~~}}$\newline\newline
		\noindent\textbf{Группа} $\underline{\text{ИУ7-63Б~~~~~~~~~~~~~~~~~~~~~}}$\newline\newline
		\noindent\textbf{Оценка (баллы)} $\underline{\text{~~~~~~~~~~~~~~~~~~~}}$\newline\newline
		\noindent\textbf{Преподаватель} $\underline{\text{Саркисян П. С.}}$\newline\newline\newline
		
		\begin{center}
			\vfill
			Москва~---~\the\year
			~г.
		\end{center}
	\end{titlepage}

\chapter*{Задание}

\section*{Цель работы}
Построение доверительных интервалов для математического ожидания и дисперсии нормальной случайной величины.

\section*{Постановка задачи}

\begin{enumerate}
	\item Для выборки объема $n$ из нормальной генеральной совокупности $X$ реализовать в виде программы на ЭВМ
	\begin{enumerate}
		\item вычисление точечных оценок $\hat\mu(\vec X_n)$ и $S^2(\vec X_n)$ математического ожидания $MX$ и дисперсии $DX$ соответственно;
		\item вычисление нижней и верхней границ $\underline\mu(\vec X_n)$, $\overline\mu(\vec X_n)$ для $\gamma$-доверительного интервала для математического ожидания $MX$;
		\item вычисление нижней и верхней границ $\underline\sigma^2(\vec X_n)$, $\overline\sigma^2(\vec X_n)$ для $\gamma$-доверительного интервала для дисперсии $DX$;
	\end{enumerate}
	\item вычислить $\hat\mu$ и $S^2$ для выборки из индивидуального варианта;
	\item для заданного пользователем уровня доверия $\gamma$ и $N$ – объёма выборки из индивидуального варианта:
	\begin{enumerate}
		\item на координатной плоскости $Oyn$ построить прямую $y = \hat\mu(\vec{x_N})$, также графики функций $y = \hat\mu(\vec x_n)$, $y = \underline\mu(\vec x_n)$ и $y = \overline\mu(\vec x_n)$ как функций объема $n$ выборки, где $n$ изменяется от 1 до $N$;
		\item на другой координатной плоскости $Ozn$ построить прямую $z = S^2(\vec{x_N})$, также графики функций $z = S^2(\vec x_n)$, $z = \underline\sigma^2(\vec x_n)$ и $z = \overline\sigma^2(\vec x_n)$ как функций объема $n$ выборки, где $n$ изменяется от 1 до $N$.
	\end{enumerate}
\end{enumerate}

\chapter*{Теоретические сведения}

\section*{Определение $\gamma$-доверительного интервала для значения параметра распределения случайной величины}

Дана случайная величина $X$, закон распределения которой известен с точностью до неизвестного параметра $\theta$.

Интервальной оценкой с уровнем доверия $\gamma$ ($\gamma$-доверительной интервальной оценкой) параметра $\theta$ называют пару статистик $\underline{\theta}(\vec X), \overline{\theta}(\vec X)$ таких, что

\begin{equation*}
	P\{\underline{\theta}(\vec X)<\theta<\overline{\theta}(\vec X)\}=\gamma
\end{equation*}

Поскольку границы интервала являются случайными величинами, то для различных реализаций случайной выборки $\vec X$ статистики $\underline{\theta}(\vec X), \overline{\theta}(\vec X)$ могут принимать различные значения.

Доверительным интервалом с уровнем доверия $\gamma$ ($\gamma$-доверительным интервалом) называют интервал $(\underline{\theta}(\vec x), \overline{\theta}(\vec x))$, отвечающий выборочным значениям статистик $\underline{\theta}(\vec X), \overline{\theta}(\vec X)$.

\section*{Формулы для вычисления границ \\ $\gamma$-доверительного интервала для математического ожидания и дисперсии нормальной случайной величины}

Формулы для вычисления границ $\gamma$-доверительного интервала для математического ожидания:

\begin{equation}
\underline\mu(\vec X_n)=\overline X + \frac{S(\vec X)t^{St(n-1)}_{\frac{1-\gamma}{2}}}{\sqrt{n}}
\end{equation}

\begin{equation}
\overline\mu(\vec X_n)=\overline X + \frac{S(\vec X)t^{St(n-1)}_{\frac{1+\gamma}{2}}}{\sqrt{n}}
\end{equation}

$\overline X$ -- точечная оценка математического ожидания;

$S(\vec X) = \sqrtsign{S^2(\vec X)}$ -- квадратный корень из точечной оценки дисперсии;

$n$ -- объем выборки;

$\gamma$ -- уровень доверия;

$t^{St(n-1)}_{\alpha}$ -- квантиль уровня $\alpha$ распределения Стьюдента с $n - 1$ степенями свободы.

Формулы для вычисления границ $\gamma$-доверительного интервала для дисперсии:

\begin{equation}
\underline\sigma(\vec X_n)= \frac{(n-1)S^2(\vec X)}{t^{\chi^2(n-1)}_{\frac{1+\gamma}{2}}}
\end{equation}

\begin{equation}
\overline\sigma(\vec X_n)= \frac{(n-1)S^2(\vec X)}{t^{\chi^2(n-1)}_{\frac{1-\gamma}{2}}}
\end{equation}

$S^2(\vec X)$ -- точечная оценка дисперсии;

$n$ -- объем выборки;

$\gamma$ -- уровень доверия;

$t^{\chi^2(n-1)}_{\alpha}$ -- квантиль уровня $\alpha$ распределения $\chi^2(n-1)$ с $n - 1$ степенями свободы.


\chapter*{Результаты работы программы}

\section*{Код программы}

\begin{lstlisting}[language=Matlab]
X = [-10.82,-9.27,-9.65,-9.36,-9.27,-11.25,-9.89,-9.26,-11.15,
	-8.90,-11.02,-8.28,-9.18,-10.16,-10.59,-10.82,-9.05,-9.47,
	-10.98,-11.50,-8.64,-10.86,-10.76,-11.49,-11.09,-9.33,
	-9.32,-9.66,-8.79,-10.54,-9.12,-10.40,-8.59,-10.22,-9.06,
	-10.59,-10.60,-10.25,-9.35,-11.44,-9.81,-9.32,-9.95,-9.33,
	-10.64,-9.45,-10.99,-10.15,-10.39,-10.36,-10.49,-11.67,
	-10.00,-10.87,-11.11,-9.68,-10.77,-9.13,-8.62,-10.33,-11.36,
	-10.24,-9.41,-11.05,-10.15,-9.35,-11.45,-9.87,-10.41,-10.11,
	-10.84,-11.48,-7.77,-10.79,-9.88,-10.70,-9.07,-9.47,-10.15,
	-9.93,-11.52,-9.04,-10.93,-10.13,-9.56,-11.39,-9.79,-9.19,
	-11.09,-9.86,-10.67,-10.26,-9.07,-10.53,-11.24,-10.16,-11.33,
	-8.76,-8.88,-10.53,-10.12,-8.98,-9.84,-9.90,-10.13,-9.32,-9.31,
	-9.99,-8.55,-11.64,-11.32,-10.51,-11.71,-10.50,-10.50,-12.20,
	-11.68,-10.45,-7.88,-10.84]

gamma = 0.9;

% 1-2
[muhat, muci] = my_normfit_mu(X, 1 - gamma);
[s2hat, s2ci] = my_normfit_s2(X, 1 - gamma);

% 3
process_mu(X, gamma, muhat);
process_s2(X, gamma, s2hat);

function [muhat, muci] = normfit_mu(X, alpha)
	[muhat, ~, muci, ~] = normfit(X, alpha);
end

function [s2hat, s2ci] = normfit_s2(X, alpha)
	[~, sigmahat, ~, sigmaci] = normfit(X, alpha);
	s2hat = sigmahat ^ 2;
	s2ci = sigmaci .^ 2;
end

function [muhat, muci] = my_normfit_mu(X, alpha)
	muhat = mean(X);
	s = std(X);
	gamma = 1 - alpha;
	n = length(X);
	mu_bottom = muhat + s * tinv((1 - gamma) / 2, n - 1) / sqrt(n);
	mu_top = muhat + s * tinv((1 + gamma) / 2, n - 1) / sqrt(n);
	muci = [mu_bottom, mu_top];
end

function [s2hat, s2ci] = my_normfit_s2(X, alpha)
	s2hat = var(X);
	gamma = 1 - alpha;
	n = length(X);
	s2_top = (n - 1) * s2hat / chi2inv((1 - gamma) / 2, n - 1);
	s2_bottom = (n - 1) * s2hat / chi2inv((1 + gamma) / 2, n - 1);
	s2ci = [s2_bottom, s2_top];
end

function process_parameter(X, gamma, est, fit, line_legend, est_legend, top_legend, bottom_legend)
	N = length(X);
	figure;
	hold on;
	grid on;
	plot([1, N], [est, est]);
	ests = [];
	cis_bottom = [];
	cis_top = [];
	for n = 1:N
		[est, cis] = fit(X(1:n), 1 - gamma);
		ests = [ests, est];
		cis_bottom = [cis_bottom, cis(1)];
		cis_top = [cis_top, cis(2)];
	end
	
	plot(1:N, ests);
	plot(1:N, cis_bottom);
	plot(1:N, cis_top);
	l = legend(line_legend, est_legend, top_legend, bottom_legend);
	set(l, 'Interpreter', 'latex', 'fontsize', 18);
	hold off;
end

function process_mu(X, gamma, muhat)
	process_parameter(X, gamma, muhat, @my_normfit_mu, '$\hat\mu(\vec x_N)$', '$\hat\mu(\vec x_n)$', '$\underline\mu(\vec x_n)$', '$\overline\mu(\vec x_n)$');
end

function process_s2(X, gamma, S2)
	process_parameter(X, gamma, S2, @my_normfit_s2, '$\hat\sigma^2(\vec x_N)$', '$\hat\sigma^2(\vec x_n)$', '$\underline\sigma^2(\vec x_n)$', '$\overline\sigma^2(\vec x_n)$');
end
\end{lstlisting}

\section*{Результаты расчётов}
\begin{equation*}
	\hat\mu(\vec x_n) = -10,1318\\
\end{equation*}

\begin{equation*}
	S^2(\vec x_n) = 0,846\\
\end{equation*}

\begin{equation*}
	\underline\mu(\vec x_n) = -10,2709\\
\end{equation*}

\begin{equation*}
	\overline\mu(\vec x_n) = -9,9926\\
\end{equation*}

\begin{equation*}
	\underline{S^2}(\vec x_n) = 0,6921\\
\end{equation*}

\begin{equation*}
	\overline{S^2}(\vec x_n) = 1,0619\\
\end{equation*}

\begin{figure}[h]
	\centering
	\includegraphics[scale=0.5]{img/1.png}1
	\caption{Прямая $y(n) = \hat\mu(\vec x_N)$, а также графики функций $y(n) = \underline\mu(\vec x_n)$, $y(n) = \overline\mu(\vec x_n)$, $y(n) = \hat\mu(\vec x_n)$ как функций объема $n$ выборки, где $n$ изменяется от 1 до $N$}
	\label{fig:1}
\end{figure}

\begin{figure}[h]
	\centering
	\includegraphics[scale=0.5]{img/2.png}
	\caption{Прямая $z(n) = \hat S^2(\vec x_N)$, а также графики функций $z(n) = \underline S^2(\vec x_n)$, $z(n) = \overline S^2(\vec x_n)$, $z(n) = \hat S^2(\vec x_n)$ как функций объема $n$ выборки, где $n$ изменяется от 1 до $N$}
	\label{fig:2}
\end{figure}

\begin{figure}[h]
	\centering
	\includegraphics[scale=0.5]{img/3.png}
	\caption{Прямая $z(n) = \hat S^2(\vec x_N)$, а также графики функций $z(n) = \underline S^2(\vec x_n)$, $z(n) = \overline S^2(\vec x_n)$, $z(n) = \hat S^2(\vec x_n)$ как функций объема $n$ выборки, где $n$ изменяется от 1 до $N$ (приближенный)}
	\label{fig:3}
\end{figure}

\bibliographystyle{utf8gost705u}  % стилевой файл для оформления по ГОСТу
\bibliography{51-biblio}          % имя библиографической базы (bib-файла)
	
\end{document}
